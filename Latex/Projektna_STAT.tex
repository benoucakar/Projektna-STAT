\documentclass[a4paper,11pt]{article}
\usepackage[slovene]{babel}
\usepackage[utf8]{inputenc}
\usepackage[T1]{fontenc}
\usepackage{lmodern}
\usepackage{amsmath, amsthm, amsfonts, amssymb}

%%%%%%%%%%%%%%%%%%%%%%%%%%%%%%%%%%%%%%%%%%%%%%%%%%%%%%%%%%%%%%%%%%%%%%%%%%%%%

\newcommand{\set}[1]{\left\{#1\right\}} % množica
\newcommand{\abs}[1]{\left|#1\right|} % absolutna vrednost

\DeclareMathOperator{\E}{E}
\DeclareMathOperator{\PP}{P}
\DeclareMathOperator{\var}{var}
\DeclareMathOperator{\Geom}{Geom}
\DeclareMathOperator{\Lver}{L}
\DeclareMathOperator{\lver}{l}

%%%%%%%%%%%%%%%%%%%%%%%%%%%%%%%%%%%%%%%%%%%%%%%%%%%%%%%%%%%%%%%%%%%%%%%%%%%%%

\begin{document}

\title{Projektna naloga pri predmetu Statistika}
\author{Beno Učakar \\ Profesor: doc.~dr.~Martin Raič}
\date{}

%%%%%%%%%%%%%%%%%%%%%%%%%%%%%%%%% - Naslov - %%%%%%%%%%%%%%%%%%%%%%%%%%%%%%%%%

\maketitle

%%%%%%%%%%%%%%%%%%%%%%%%%%%%%%%%% - 1. naloga - %%%%%%%%%%%%%%%%%%%%%%%%%%%%%%%%%



%%%%%%%%%%%%%%%%%%%%%%%%%%%%%%%%% - 2. naloga - %%%%%%%%%%%%%%%%%%%%%%%%%%%%%%%%%

\section{2. naloga}

\subsection{Primer (a)}

Najprej uvedimo nekaj oznak. 
Če je $k \in \set{1, \ldots, 12}$ število skokov ptic, naj bo $S_k$ frekvenca tega opažanja.
Podatke z novimi oznakami predstavimo v spodnji tabeli.
\begin{table}[h]
    \centering
    \begin{tabular}{|l|l|l|l|l|l|l|l|l|l|l|l|l|}
    \hline
    $k$ & 1 & 2 & 3 & 4 & 5 & 6 & 7 & 8 & 9 & 10 & 11 & 12 \\ \hline
    $S_k$ & 48 & 31 & 20 & 9 & 6 & 5 & 4 & 2 & 1 & 1 & 2 & 1 \\ \hline
    \end{tabular}
    \caption{Frekvence števila skokov}
    \label{freq}
\end{table}
Skupno število opaženih skokov označimo z $S$, število vseh opažanj pa z $N$.
Velja
\[N = \sum_{k=1}^{12} S_k \qquad S = \sum_{k=1}^{12} k S_k.\]
V našem primeru znaša $N = 130$ in $S = 363$.

Želimo poiskati geometrijsko porazdelitev, ki se najbolje prilega tem podatkom. 
Če je število skokov posameznega ptiča slučajna spremenljivka $K \sim \Geom(p)$, 
v resnici iščemo cenilko za parameter $p$.
To znamo narediti na vsaj dva načina.

\emph{Prvi način:} Postopamo po metodi momentov. 
Spomnimo se, da pričakovana vrednost geometrijske porazdelitve $\Geom(p)$ znaša $\frac{1}{p}$.
Zato velja
\[p = \frac{1}{E(K)}.\]
Po metodi momentov $\E(K)$ ocenimo z 
\[\frac{1}{N} \sum_{k=1}^{12} k S_k = \frac{S}{N},\]
kar nam da cenilko
\[\hat{p} = \frac{N}{S}.\]

\emph{Drugi način:} Postopamo po metodi največjega verjetja.
Verjetnostna funkcija geometrijske porazdelitve $\Geom(p)$ je $\PP(K=k) = p(1-p)^{k-1}$.
Verjetje lahko torej izrazimo kot 
\[\Lver(p \mid S1, \ldots, S_{12}) = p^N (1-p)^{S - N}.\]
Ko logaritmiramo, dobimo 
\[\lver(p \mid S1, \ldots, S_{12}) = N \ln(\frac{p}{1-p}) + S \ln(1-p).\]
Če parcialno odvajamo po $p$ in malo računamo, ponovno pridemo do cenilke
\[\hat{p} = \frac{N}{S}.\]

V obeh primerih pridemo do iste cenilke.
Ta v našem primeru znaša 
\[\hat{p} = \frac{130}{363} \approx 0.358.\]




%%%%%%%%%%%%%%%%%%%%%%%%%%%%%%%%% - 3. naloga - %%%%%%%%%%%%%%%%%%%%%%%%%%%%%%%%%

\end{document}